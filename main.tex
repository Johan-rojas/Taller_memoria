\documentclass{article}
\usepackage[utf8]{inputenc}
\usepackage[spanish]{babel}
\usepackage{listings}
\usepackage{graphicx}
\graphicspath{ {images/} }
\usepackage{cite}

\begin{document}

\begin{titlepage}
    \begin{center}
        \vspace*{1cm}
            
        \Huge
        \textbf{Taller Memoria}
            
        \vspace{0.5cm}
        \LARGE
        Taller - Nociones de la memoria del computador
            
        \vspace{1.5cm}
            
        \textbf{Johan David Rojas Martinez}
            
        \vfill
            
        \vspace{0.8cm}
       
        \Large
\begin{figure}[h]
\includegraphics[width=4cm]{logoudea.png}
\centering
\end{figure}

        \vfill
        Despartamento de Ingeniería Electrónica y Telecomunicaciones\\
        Universidad de Antioquia\\
        Medellín\\
        Septiembre de 2020
                 
    \end{center}
\end{titlepage}

\tableofcontents

\section{Introduccion}
Esta es la primera sección, podemos agregar algunos elementos adicionales y todo será escrito correctamente. 

\section{Preguntas} \label{contenido}

\subsection{Defina que es la memoria del computador.}
\subsection{Mencione los tipos de memoria que conoce y haga una pequeña descripción de cada tipo.}
\subsubsection{Memoria caché}
\subsubsection{Memoria RAM}
\subsubsection{Memoria Virtual}
\subsubsection{Disco duro}

\subsection{Describa la manera como se gestiona la memoria en un computador.}
\subsection{¿Qué hace que una memoria sea más rápida que otra? ¿Por qué esto es importante?}

\section{Conclusión} \label{conclulsion}

\bibliographystyle{IEEEtran}
\bibliography{references}

\end{document}
